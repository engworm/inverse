% \RequirePackage{plautopatch}
\documentclass[uplatex,12pt]{jsarticle}
\usepackage{listings,jvlisting}
\usepackage[T1]{fontenc}
\usepackage{lmodern}
\usepackage{empheq}
\usepackage{ulem}
\usepackage{amsmath, amsthm}
\usepackage{mathtools}
\usepackage{cases}
\usepackage{ascmac}
\usepackage{dirtree}
\usepackage{bbm}
\usepackage{ulem}
\usepackage[svgnames]{xcolor}

\begin{document}

\section{Title}

Today I would like to talk about my master paper.
The title is ``A New Setting for Some Inverse Potential Problem and The Bubbling Method."

\section{Abstraction}

Here is a figure of core-shell model, this is a simple model of a planet.
In the simple model, from outside information, we can deduce something internal information, for example, the shape of the core.
For example, when we chase satellite and observe its orbit, we can compute gravity.
If gravity we observe is pulling this satellite strong, 
this means something heavy is buried directly below the satellite.
This is why we can deduce the shape of the core.
In this way, Gravity has an important role to deduce the internal stracture of a planet.

However, Gravity measurement might be outdated.
Today, we can observe the gravitational potential from atomic clock directly!
The potential is more rubust against the distance of target to observer.
% Note that Observer is kind of a satellite.
So, we can expect that potential observation has more information about internal structure than gravity's.

This first line is the main topic of my research.
``Can we expect that potential takes the place of the gravity?''

\section{Core-Shell Body}

We introduce the inverse potential problem in gravimetry for core-shell body.
Please notice the figure below to your left.
On dark shadow domain there is density with $\rho +1$,
on light shadow domain there is density with $1$.
Potential is
\[
  U = U^{B_R}+\rho U^{\Omega},
\]
where E is the fundamental solution of the Laplace equation.
Move right hand side's first term to left and change side, we get
\[
  \rho U^{\Omega} = U-U^{B_R}.
\]
$U$ is observation value of $\partial B_a$ and $U^{B_R}$ can be calculated.
So, we can get the value of $U^\Omega$ on observation surface $\partial B_a$.
This is corresponding to the figure bottom right.
Hereafter we set problem on this figure.

\section{New Setting for The Inverse Potential Problem}

In this slide, we will show you problem setting for the traditional inverse potential problem in gravimetry.
traditionaly, the problem setting is that when we observe gravity on $\partial B_a$,
reconstruct the shape of $\Omega$.

Today, potential observation has became possible by atomic clock.
So, we can replace this problem settings to new ones.
Boundary condition is known as potential observation.
It is plausable to think that potential has more information about internal structure than gravity has.
This is why we expect potential observation generates better reconstruction shape of the core.

\section{Sketch of Reconstruction Algorithm}

I would like to explain how to reconstruct the shape of $\Omega$ by computation.
Actually, there are finite points we can observe.
We change previous boundary condition to following discrete version boundary condition.

Reconstruction algorithm consits of two parts.
In the first stage, approximates source body by a set of point mass.
In the second stage, homogenize ths set we obtained in the previous stage.
Each stage is implemented by optimization method and bubbling method.

\section{Sketch of Reconstruction Algorithm (1)}

Next slide, I will explain the first part of the reconstruction algorithm more specifically.

\section{Apprx.~Body by a Set of Point Masses (Gravity observation)}

When we observe gravity at $\{A_n\}_{n=1}^N$, we define cost function $J_G$ as following.
The first term means observed value of gravity, the second term means gravity generated by a set of point mass.

\section{Apprx.~Body by a Set of Point Masses (Potential observation)}

We do the same thing in previous slide.
When we obseve potential at observation points, we define cost function $J_P$ as following.
The first term means observed value of potential, the second term means potential generated by a set of point mass.

\section{Sketch of Reconstruction Algorithm (2)}

Next slide, I will explain the second stage.

\section{Bubbling Method (Partial Mass Scattering)}

In this part, we homogenize a set of point mass to homogeneous body with density $\rho$.
Please notice the figure placed in bottom left.
Firstly, point mass we get by the first stage move to the most closed lattice point.
Secondly, please look the figure below to your right, if the point mass beyound this threshold, 
homogenize and flow equally their own mass to the closest four lattice points following by the procedure.
Repeatedly do this procedure until nothing mass flows occur.
Finally, we can get homogeneous body with density about $\rho$.

\section{Reconfirm The Aid of Computation}

We are ready to compute the inverse potential problems in gravimetry.
Before start computing, we should reconfirm our aim.
Keyword is ``Can we expect the potential takes the place of the gravity?''
This keyword cames from the toughness against the distance of target to observer.
When observation radius $a$ bigger, what will be happened?
When we change radius of observation $a$, 
we verify influence for the reconstruction which is caused by this change.

\section{Example: Reconstruction of Ellipzoid}

Set $R=2$.
We reconstruct an elliptic shape with long radius $\sqrt{2}$, short radius $1$ and density $\rho=10$.
Limit of resolution is $10^{-4}$.

We select Levenberg-Marquardt method as a optimization method.

\section{Example: Gravity Observation}

Upper left ... 
Upper right ... 
Bottom left ...
Bottom right ...


\section{Example: Potential Observation}

Upper left ... 
Upper right ... 
Bottom left ...
Bottom right ...

\section{Conclusion}

We observe the potential and reconstruct the shape of the body, compared to observation of the gravity.
Keyword is ``Can we expect the potential takes the place of the gravity?''
In the case of reconstruction of ellipzoid by bubbling method, the answer is "yes."

\end{document}
